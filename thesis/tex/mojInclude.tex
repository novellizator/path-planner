\newtheorem{theorem}{Veta}
\newtheorem{define}{Definícia}	
\newtheorem{note}{Poznámka}
\newtheorem{example}{Príklad}
\newtheorem{consequence}{Dôsledok}


\usepackage{amsfonts} % matematika
%\usepackage{graphicx} % svg grafika % zbytocne???

\usepackage{graphicx}

% 5 kvoli velkemu Ocku pre casovu zlozitost
\usepackage{amsmath}
\usepackage{amssymb}
\usepackage{mathtools}

\newcommand{\BigO}[1]{\ensuremath{\operatorname{O}\bigl(#1\bigr)}}


% mnoziny
\newcommand{\N}{\mathbb{N}\,}
\newcommand{\Z}{\mathbb{Z}\,}
\newcommand{\Q}{\mathbb{Q}\,}
\newcommand{\R}{\mathbb{R}\,}
\newcommand{\Ri}{\mathbb{R}^{*}\,}


\usepackage{listings} % zdrojaky c++
\usepackage{color}
\usepackage{xcolor}

\lstset { %
belowcaptionskip=1\baselineskip,
breaklines=true,
language=C++,
showstringspaces=false,
basicstyle=\footnotesize\ttfamily,
keywordstyle=\bfseries\color{green!40!black},
commentstyle=\itshape\color{purple!40!black},
identifierstyle=\color{blue},
stringstyle=\color{orange},
backgroundcolor=\color{black!5}
}


%\usepackage[linesnumbered,ruled,vlined]{algorithm2e}
\usepackage{algorithm}
\usepackage{algorithmic} % http://ftp.cvut.cz/tex-archive/macros/latex/contrib/algorithms/algorithms.pdf
\floatname{algorithm}{Algoritmus}
\renewcommand{\algorithmicrequire}{\textbf{Vstup:}}
\renewcommand{\algorithmicensure}{\textbf{Výstup:}}
\renewcommand{\algorithmiccomment}[1]{\{#1\}}


\usepackage[toc,page]{appendix}
\renewcommand{\appendixtocname}{Prílohy}
\renewcommand\appendixname{Príloha}
\renewcommand\appendixpagename{Prílohy}
  