\newtheorem{theorem}{Veta}
\newtheorem{define}{Definícia}	
\newtheorem{note}{Poznámka}
\newtheorem{example}{Príklad}
\newtheorem{consequence}{Dôsledok}


\usepackage{amsfonts} % matematika
%\usepackage{graphicx} % svg grafika % zbytocne???

\usepackage{graphicx}

% 5 kvoli velkemu Ocku pre casovu zlozitost
\usepackage{amsmath}
\usepackage{amssymb}
\usepackage{mathtools}

\newcommand{\BigO}[1]{\ensuremath{\operatorname{O}\bigl(#1\bigr)}}


% mnoziny
\newcommand{\N}{\mathbb{N}\,}
\newcommand{\Z}{\mathbb{Z}\,}
\newcommand{\Q}{\mathbb{Q}\,}
\newcommand{\R}{\mathbb{R}\,}
\newcommand{\Ri}{\mathbb{R}^{*}\,}


\usepackage{listings} % zdrojaky c++
\usepackage{color}
%\lstset{ %
%language=C++,                % choose the language of the code
%basicstyle=\footnotesize,       % the size of the fonts that are used for the code
%numbers=left,                   % where to put the line-numbers
%numberstyle=\footnotesize,      % the size of the fonts that are used for the line-numbers
%stepnumber=1,                   % the step between two line-numbers. If it is 1 each line will be numbered
%numbersep=5pt,                  % how far the line-numbers are from the code
%backgroundcolor=\color{white},  % choose the background color. You must add \usepackage{color}
%showspaces=false,               % show spaces adding particular underscores
%showstringspaces=false,         % underline spaces within strings
%showtabs=false,                 % show tabs within strings adding particular underscores
%frame=single,           % adds a frame around the code
%framesep=15pt,
%tabsize=2,          % sets default tabsize to 2 spaces
%captionpos=b,           % sets the caption-position to bottom
%breaklines=true,        % sets automatic line breaking
%breakatwhitespace=false,    % sets if automatic breaks should only happen at whitespace
%escapeinside={\%*}{*)}          % if you want to add a comment within your code
%}

\lstset { %
belowcaptionskip=1\baselineskip,
breaklines=true,
language=C++,
showstringspaces=false,
basicstyle=\footnotesize\ttfamily,
keywordstyle=\bfseries\color{green!40!black},
commentstyle=\itshape\color{purple!40!black},
identifierstyle=\color{blue},
stringstyle=\color{orange},
backgroundcolor=\color{black!5}
}


%\usepackage[linesnumbered,ruled,vlined]{algorithm2e}
\usepackage{algorithm}
\usepackage{algorithmic} % http://ftp.cvut.cz/tex-archive/macros/latex/contrib/algorithms/algorithms.pdf
\floatname{algorithm}{Algoritmus}
\renewcommand{\algorithmicrequire}{\textbf{Vstup:}}
\renewcommand{\algorithmicensure}{\textbf{Výstup:}}
\renewcommand{\algorithmiccomment}[1]{\{#1\}}



% \hyphenation{naj-d{\^o}-le-zi-tej-sich}
