\chapter{Súťažný algoritmus}
Súťažný algoritmus bude využívať poznatky popísané v predošlej kapitole. Navyše zavedieme koncept tzv. {\sl mriežkového grafu bez prekážok}, ktorý umožní mierne zrýchliť výkon algoritmu v mnohých prípadoch, väčšinou však pri trasách, kde počiatočný a koncový bod ležia relatívne \uv{blízko seba}.


\section{Mriežkový graf bez prekážok}
Nie všetky najkratšie cesty musia obchádzať veľa prekážok. 
V mnohých prípadoch neleží medzi počiatočným a koncovým bodom žiadna prekážka, a~teda cesty sú veľmi priamočiare. To sa pokúsime využiť na~zlepšenie výkonu algoritmu.
Pre ľahšie vyjadrovanie si zaveďme definíciu {\sl mriežkového grafu bez prekážok}.

\begin{define}
{\sl Mriežkový graf je bez prekážok}, 
pokiaľ medzi každými dvoma susednými vrcholmi existuje hrana.
\end{define}
Kvôli lepšej prehľadnosti a stručnosti budeme mriežkový graf bez prekážok nazývať aj {\sl obdĺžnik}.
Intuitívne, kvôli jeho vizuálnej predstave. Jeho obsahom bude počet vrcholov patriacich do tohto obdĺžniku.

Majme mriežkový graf bez~prekážok a~hľadajme najkratšiu cestu medzi bodmi $s=(x_s,y_s), t=(x_t,y_t)$.
V~tomto prípade vieme nájsť najkratšiu cestu veľmi jednoducho.

\begin{algorithm}
\caption{Nájdi najkratšiu cestu medzi dvoma bodmi $s$ a $t$ na mriežkovom grafe bez prekážok}
\label{alg1}
\begin{algorithmic}[1] % number one = line numbering is on
\REQUIRE $s=(x_s,y_s), t=(x_t,y_t)$
\ENSURE $path$


\STATE path.append($(x_s, y_s)$)
\COMMENT {pridám počiatok}

\WHILE {$x_s \neq x_t \vee y_s \neq y_t $}
	\IF {$x_s < x_t$}
		\STATE $x_s \leftarrow x_s + 1$
	\ELSIF {$x_s > x_t$}
		\STATE $x_s \leftarrow x_s - 1$
	\ENDIF

	\IF {$y_s < y_t$}
		\STATE $y_s \leftarrow y_s + 1$
	\ELSIF {$y_s>y_t$}
		\STATE $y_s \leftarrow y_s - 1$
	\ENDIF
	\STATE path.append($(x_s, y_s)$)
\ENDWHILE

\end{algorithmic}
\end{algorithm}



Jednoducho povedané: keď sa počiatočný a koncový bod líšia v jednej súradnici, tak sa posúvame priamočiaro,
keď sa líšia v oboch, tak sa posúvame šikmo.


Pokiaľ si zadefinujeme 
$ dx := |x_t - x_s|$ 
a
$ dy :=|y_t - y_s| $
 , tak počet vrcholov,
ktorými cesta prechádza vieme zhora odhadnúť, ako $\max(dx, dy)$. Jej vzdialenosť vieme zistiť v čase  $\BigO{\max(dx, dy)}$
Na zistenie vzdialenosti v každom kroku nám postačí konštantná pamäť.


Pokiaľ sa počiatočný a koncový bod cesty nachádza v jednom obdĺžniku, tak vieme pomocou tohto algoritmu veľmi rýchlo nájsť najkratšiu cestu.
Jediným problémom ostalo rozdeliť mriežkový graf na tieto obdĺžniky. 


\section{Hľadáme obdĺžniky}
Našou snahou je dekomponovať mriežkový graf do obdĺžnikov tak, aby bola čo najväčšia pravdepodobnosť toho, že 
dva náhodne zvolené body (začiatok a koniec) ležia v jednom obdĺžniku. Čiže aby sa dalo toto zrýchlené
hľadanie najkratšej cesty použiť v najväčšom možnom počte prípadov.


\subsection{Proporcie obdĺžnikov}

Dôležitou otázkou je, na akých vlastnostiach obdĺžnikov záleží. Uvažujme nasledujúci motivačný príklad.
\begin{example}
Majme na mriežkovom grafe nájdené dva obdĺžniky, ktoré dovedna pokrývajú 10 vrcholov.
Predstavme si tieto dva prípady. V prvom prípade prvý pokrýva 9 vrcholov, druhý 1. V druhom prípade obdĺžniky pokrývajú 6 vrcholov a 4 vrcholy.
Chceme maximalizovať pravdepodobnosť toho, aby pri voľbe dvoch náhodných bodov boli oba body v rovnakom obdĺžniku.
\end{example}

Úlohu vieme zobecniť na klasickú pravdepodobnostno-optimalizačnú úlohu.

\begin{example}
Máme $k$ ekvivalenčných tried na množine s $n$ prvkami. Ako zvoliť ekvivalenčné triedy tak, 
aby pri voľbe dvoch náhodných prvkov bola pravdepodobnosť toho, 
že oba prvky budú v tej istej ekvivalenčnej triede čo najvyššia?
\end{example}

\begin{note}
Ekvivalenčnú triedu predstavuje obdĺžnik a množinu predstavuje množina vrcholov grafu.
Alternatívne sa môžeme na úlohu pozerať ako na problém farbenia $n$ guličiek pomocou $k$ farieb.
\end{note}


Zapíšme túto úlohu formálne. Majme $n$-prvkovú množinu 
\[Prv = \{x_1,\ldots,x_n\}\] $k$-prvkovú množinu ekvivalenčných tried \[Ek = \{ek_1,\ldots, ek_k\}.\] veľkosť 
triedy $\|ek_i\|$ označme $k_i$ a zaveďme funkciu $f \colon Prv \to Ek$ ktorá roztriedi prvky do ekvivalenčných tried.

Označme výberový priestor \[\Omega = \{(x_a, x_b) | x_a, x_b \in Prv, a \not= b \}.\]
Udalosťou $A_i$ nazveme jav, v~ktorom oba prvky patria do tej istej ekvivalenčnej triedy $ek_i$,
teda \[A_i = \{(x_a, x_b) | x_a, x_b \in Prv, a \not= b, f(x_a) = f(x_b) = ek_i \}.\]
Jav \[A = \bigcup_{i=1}^{k} A_i\] teda nastáva práve vtedy,
 keď oba vybrané prvky patria do rovnakej triedy.

Úlohou je navrhnúť funkciu $f$ tak, aby pravdepodobnosť $P[A]$ bola čo najvyššia. 
Keďže udalosti $A_i$ sú nezlučiteľné, môžme písať 
\[P[A] = P[\bigcup_{i \in Ek} A_i] = \sum_{i \in Ek}P[A_i].\]

Ak si pravdepodobnosť každého javu rozpíšeme, dostaneme
\[\sum_{i \in Ek}P[A_i] = \sum_{i = 1}^{k} \frac{{{k_i} \choose {2}}}{{{|Prv|} \choose {2}}}.\] 



Keďže chceme nejak rozvrhnúť prvky v triedach $ek_i$, a menovateľ je konštanta, môžme ho vynechať.

Maximalizujeme hodnotu výrazu 
\[\sum_{i = 1}^{k} {{k_i} \choose {2}} = \sum_{i = 1}^{k} {\frac{k_i!}{(k_i -2 )!2!}} = \sum_{i = 1}^{k}{\frac{k_i (k_i-1)}{2}}.\]



Po vyškrtnutí konštanty a roznásobení sme dostali nasledujúcu optimalizačnú úlohu:
maximalizovať \[\sum_{i = 1}^{k} {k_i^2 - k_i}\] za podmienok \[\sum_{k=1}^{k}k_i = n,k_i \in \N_0.\]


Sumu si vieme rozpísať ako 
\[\sum_{i = 1}^{k} {k_i^2 - k_i} = \sum_{i = 1}^{k} {k_i^2} + \sum_{i = 1}^{k}{-k_i}.\]
Druhá suma sa nasčíta na $-n$, čo je konštanta, takže nám stačí maximalizovať 
\[\sum_{i = 1}^{k} {k_i^2}.\]

Teraz nám už len zostáva použiť nerovnosť
\[(a+b)^2 \geq a^2 + b^2,\] ktorá platí pre $a,b \geq 0$, z ktorej jasne vyplýva, že potrebujeme spraviť ľubovoľné $k_i$ čo najväčšie
a teda aj ekvivalenčné triedy musia byť čo najväčšie.

Potrebujeme teda nájsť obdĺžniky s~najväčším možným obsahom. 
V~programe tento problém rieši trieda Colorizator.
Tá v grafe nájde najväčší obdĺžnik a \uv{ofarbí} ho jednou farbou.
Na zvyšnom neofarbenom grafe znovu hľadá najväčší obdĺžnik a ofarbí ho inou farbou, atď.
Farbenie skončí, keď obsah najväčsieho obdĺžnika je menší ako 500.


\subsection{Nájdenie najväčšej jednotkovej podmatice}
Ako sme si v~úvode povedali, mriežkovú mapu vieme reprezentovať ako maticu, čiže
problém môžeme ekvivalentne zapísať ako problém hľadania najväčšej jednotkovej podmatice.
Tento problém má riešenie v~čase lineárnom od~počtu vrcholov, takže nájdenie $k$ najväčších
jednotkových matíc trvá $\BigO{k \cdot n}$, kde $n$ je počet vrcholov matice.

Algoritmus je popísaný v blogposte od A. Agrawala \cite{agrawal_largest_submatrix}.
Slovný popis algoritmu \ref{alg:largest_submatrix}: 
V~prvom prechode maticou si u~každého vrcholu zapamätáme počet jedničiek naľavo od neho, vrátane daného vrcholu.  
Tento počet jedničiek môžme nazvať aj výškou riadku vzhľadom k danému vrcholu. Výšku riadku vzhľadom k $p$ označujeme v programe $nalavoOdPrvku(p)$.
Tento prechod trvá lineárny čas. 

V~druhom prechode algoritmus prechádza zaradom všetky stĺpce zľava doprava. V prechádzanom stĺpci prejde každý vrchol $v$ a zisťuje počet vrcholov nad (uloží do $DlzkaSekvencieNahor$) a pod ($DlzkaSekvencieNadol$) vrcholom $v$, ktoré majú výšku riadku aspoň $nalavoOdPrvku(v)$. 

Obsah najväčšej podmatice vzhľadom k $v$ teda vypočítame ako\\
$ nalavoOdPrvku(v) \cdot (sequenceSizeUp + 1 + sequenceSizeDown)$.



\begin{algorithm}
\caption{Nájdenie najväčšej jednotkovej podmatice v matici  $m$x$n$}
\label{alg:largest_submatrix}
\begin{algorithmic}[1] % number one = line numbering is on
\REQUIRE matica $M$ rozmerov $m \times n$ nad telesom $\Z_2$
\ENSURE polia DlzkaSekvencieNadol a DlzkaSekvencieNahor, z ktorých vieme vypočítať veľkosť podmatice


\FORALL {prvok $p$, $p \in M$}
	\IF {$p = 0$}
		\STATE nalavoOdPrvku(p) $\leftarrow$ 0
	\ELSE
		\STATE nalavoOdPrvku(p) $\leftarrow$ najdlhšia súvislá postupnosť jednotiek končiaca prvkom $p$
	\ENDIF	
\ENDFOR

\FOR {stĺpec $s$, $s \in M$ }
	\STATE Vytvor nový zásobník na dvojice čísel (riadok, nalavoOdPrvku)

	\STATE // prejdem stĺpec zhora nadol
	\FOR {riadok $r$, $r \in M$, prechádzaný zhora nadol}
		\STATE $p \leftarrow (s, r)$
		
		\WHILE {Zásobník je neprázdny}
			\STATE vyberiem prvok $top$ z vrcholu zásobníka
			\IF {$top$.nalavoOdPrvku $>$ nalavoOdPrvku(p)}
						
				\STATE prvokZoZasobnika $\leftarrow$ ($top$.r, s)
				\STATE DlzkaSekvencieNadol(prvokZoZasobnika) = r - prvokZoZasobnika.r -1
			
			\ELSE
				\STATE Vložím prvok $prvokZoZasobnika$ do zásobníka
				\STATE break
			\ENDIF
		\ENDWHILE
		
		\STATE Do zásobníka vložím dvojicu (r, nalavoOdPrvku(p))
	\ENDFOR
	
	\STATE // prešli sme stĺpec, teda skončíme
	\WHILE {Zásobník je neprázdny}
		\STATE vyberiem prvok $top$ z vrcholu zásobníka
		\STATE prvokZoZasobnika $\leftarrow$ ($top$.r, s)
		\STATE DlzkaSekvencieNadol(prvokZoZasobnika) = m - prvokZoZasobnika.r - 1
		
	\ENDWHILE

	
\ENDFOR




\end{algorithmic}
\end{algorithm}


\begin{algorithm}

\caption{pokračovanie}
\begin{algorithmic}

	
	
	\STATE // prejdem stĺpec zdola nahor

	\FOR {riadok $r$, $r \in M$, prechádzaný zdola nahor }
		\STATE $p \leftarrow (s, r)$
		
		\WHILE {Zásobník je neprázdny}
			\STATE vyberiem prvok $top$ z vrcholu zásobníka
			\IF {$top$.nalavoOdPrvku $>$ nalavoOdPrvku(p)}
						
				\STATE prvokZoZasobnika $\leftarrow$ ($top$.r, s)
				\STATE DlzkaSekvencieNahor(prvokZoZasobnika) = prvokZoZasobnika.r - r - 1
			
			\ELSE
				\STATE Vložím prvok $prvokZoZasobnika$ do zásobníka
				\STATE break
			\ENDIF
		\ENDWHILE
		
		\STATE Do zásobníka vložím dvojicu (r, nalavoOdPrvku(p))
	\ENDFOR
	
	\STATE // prešli sme stĺpec, teda skončíme
	\WHILE {Zásobník je neprázdny}
		\STATE vyberiem prvok $top$ z vrcholu zásobníka
		\STATE prvokZoZasobnika $\leftarrow$ ($top$.r, s)
		\STATE DlzkaSekvencieNahor(prvokZoZasobnika) = prvokZoZasobnika.r
		
	\ENDWHILE


\end{algorithmic}
\end{algorithm}