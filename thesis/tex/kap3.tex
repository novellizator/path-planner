\chapter{Nový algoritmus: NovellA*}

\section{Zlepšenie výkonu v niektorých prípadoch}
Nie všetky najkratšie cesty ale musia obchádzať veľa prekážok. 
V mnohých prípadoch nemusí medzi počiatočným a koncovým bodom ležať žiadna prekážka, a teda cesty sú veľmi priamočiare. To sa pokúsime využiť na~zlepšenie výkonu algoritmu.
 
Pre ľahšie vyjadrovanie si zaveďme definíciu {\sl mriežkového grafu bez prekážok}.

\begin{define}
{\sl Mriežkový graf je bez prekážok} 
pokiaľ medzi každými dvoma susednými vrcholmi existuje hrana.
\end{define}


Kvôli lepšej prehľadnosti budeme mriežkový graf bez prekážok nazývať aj {\sl obdĺžnik}.
Intuitívne, kvôli jeho vizuálnej interpretácii.

Majme mriežkový graf bez prekážok a hľadajme najkratšiu cestu medzi bodmi $s=(x_s,y_s), t=(x_t,y_t)$.
V tomto prípade vieme nájsť najkratšiu cestu veľmi jednoducho.

\begin{algorithm}
\caption{Nájdi najkratšiu cestu medzi dvoma bodmi {\sl s} a {\sl t} na mriežkovom grafe bez prekážok}
\label{alg1}
\begin{algorithmic}[1] % number one = line numbering is on
\REQUIRE $s=(x_s,y_s), t=(x_t,y_t)$
\ENSURE $path$


\STATE path.append($(x_s, y_s)$)
\COMMENT {pridám počiatok}

\WHILE {$x_s \neq x_t \vee y_s \neq y_t $}
	\IF {$x_s < x_t$}
		\STATE $x_s \leftarrow x_s + 1$
	\ELSIF {$x_s > x_t$}
		\STATE $x_s \leftarrow x_s - 1$
	\ENDIF

	\IF {$y_s < y_t$}
		\STATE $y_s \leftarrow y_s + 1$
	\ELSIF {$y_s>y_t$}
		\STATE $y_s \leftarrow y_s - 1$
	\ENDIF
	\STATE path.append($(x_s, y_s)$)
\ENDWHILE

\end{algorithmic}
\end{algorithm}



Teda, jednoducho povedané: keď sa počiatočný a koncový bod líšia v jednej súradnici, tak sa posúvame priamočiaro,
keď sa líšia v oboch, tak sa posúvame šikmo.


Pokiaľ si zadefinujeme 
$ dx := |x_t - x_s|$ 
a
$ dy :=|y_t - y_s| $
 , tak počet vrcholov,
ktorými cesta prechádza vieme zhora odhadnúť, ako $\max(dx, dy)$. Jej vzdialenosť vieme zistiť v čase  $\BigO{\max(dx, dy)}$
Na zistenie vzdialenosti v každom kroku nám stači konštantná pamäť.

TODO?? Poznamka, ze nepotrebujem na to obdlzniky, mozem robit aj komplikovanejsie utvary, ale by to sa blbo hladalo... ledaze...

Skúsme to teda nejak využiť. Pokiaľ vieme, že počiatočný aj koncový bod ležia v jednom obdĺžniku, tak máme problém vyriešený. 
Jediným problémom ostalo takéto obdĺžniky nájsť. 


\section{Hľadáme obdĺžniky}


\subsection{Proporcie obdĺžnikov}
Dôležitou otázkou je, na akých vlastnostiach obdĺžnikov záleží. Uvažujme nasledujúci motivačný príklad.
\begin{example}
Majme na mape nájdené dva obdĺžniky, ktorých celkový obsah je 10.
Predstavme si tieto dva prípady. V prvom prípade je obsah prvého 9 a druhého 1, v druhom prípade sú obsahy 6 a 4. 
Chceme maximalizovať pravdepodobnosť toho, aby pri voľbe dvoch náhodných bodov boli obe v rovnakom obdĺžniku.
\end{example}


Úlohu vieme zobecniť na klasickú pravdepodobnostno-optimalizačnú úlohu.

\begin{example}
Majme {\sl k} ekvivalenčných tried na množine s {\sl n} prvkami. Ako zvoliť ekvivalenčné triedy tak, 
aby pri voľbe dvoch náhodných prvkov bola pravdepodobnosť toho, 
že oba prvky budú v tej istej ekvivalenčnej triede čo najvyššia?
\end{example}

\begin{note}
Ekvivalenčnú triedu predstavuje obdĺžnik a množinu predstavuje množina vrcholov grafu.
Alternatívne sa môžeme na úlohu pozerať ako na problém farbenia guličiek čo najmenším počtom farieb.
\end{note}

Zapíšme túto úlohu formálne.


Majme $n$-prvkovú množinu $Prv = \{x_1,\ldots,x_n\}$, $k$-prvkovú množinu ekvivalenčných tried $Ek = \{ek_1,\ldots, ek_k\}$, veľkosť 
triedy $\|ek_i\|$ označme $k_i$ a zaveďme funkciu $f \colon Prv \to Ek$ ktorá roztriedi prvky do ekvivalenčných tried.

Označme výberový priestor $\Omega = \{(x_a, x_b) | x_a, x_b \in Prv, a \not= b \}$
Udalosťou $A_i$ nazveme jav, v~ktorom oba prvky patria do tej istej ekvivalenčnej triedy $ek_i$,
teda $A_i = \{(x_a, x_b) | x_a, x_b \in Prv, a \not= b, f(x_a) = f(x_b) = ek_i \}$
Jav $A = \bigcup_{i=1}^{k} A_i$ teda nastáva práve vtedy,
 keď oba vybrané prvky patria do rovnakej triedy.

Úlohou je teda navrhnúť funkciu $f$ tak, aby pravdepodobnosť $P[A]$ bola čo najvyššia. 
Keďže udalosti $A_i$ sú nezlučiteľné, môžme písať 
$P[A] = P[\bigcup_{i \in Ek} A_i] = \sum_{i \in Ek}P[A_i]$.

Ak si pravdepodobnosť každého javu rozpíšeme, dostaneme 
$\sum_{i \in Ek}P[A_i] = \sum_{i = 1}^{k} \frac{{{k_i} \choose {2}}}{{{|Prv|} \choose {2}}}$.


nakoľko chceme nejak rozvrhnúť prvky v triedach $ek_i$, a menovateľ je len
konštanta, môžme ho vynechať.

Maximalizujeme teda hodnotu výrazu 
$\sum_{i = 1}^{k} {{k_i} \choose {2}} = \sum_{i = 1}^{k} {\frac{k_i!}{(k_i -2 )!2!}} = \sum_{i = 1}^{k}{\frac{k_i (k_i-1)}{2}}$.
Po vyškrtnutí konštanty a roznásobení sme dostali nasledujúcu optimalizačnú úlohu:
maximalizovať $\sum_{i = 1}^{k} {k_i^2 - k_i}$ za podmienok $\sum_{k=1}^{k}k_i = n$,
kde $k_i \in \N_0$.

Sumu si vieme rozpísať ako 
$\sum_{i = 1}^{k} {k_i^2 - k_i} = \sum_{i = 1}^{k} {k_i^2} + \sum_{i = 1}^{k}{-k_i}$
druhá suma sa nasčíta $-n$, čo je konštanta, takže nám stačí maximalizovať 
$\sum_{i = 1}^{k} {k_i^2}$.

Teraz nám už len zostáva použiť nerovnosť
$(a+b)^2 \geq a^2 + b^2$ ktorá platí pre $a,b \geq 0$, z ktorej jasne vyplýva, že potrebujeme spraviť ľubovoľné $k_i$ čo najväčšie.
Ekvivalenčné triedy musia teda byť čo najväčšie a problém sa transformuje na problém hľadania
obdĺžnikov s najväčším možným obsahom.
V programe tento problém rieši trieda Colorizator.


\subsection{Nájdenie najväčšej jednotkovej podmatice}
Ako sme si v úvode povedali, mriežkovú mapu vieme reprezentovať ako maticu a teda
problém môžeme ekvivalentne zapísať ako problém hľadania najväčšej jednotkovej podmatice.
Tento problém má riešenie v čase lineárnom od počtu vrcholov a teda nájdenie $k$ najväčších
jednotkových matíc trvá $\BigO{k*n}$, kde n je počet vrcholov matice.

Slovný popis algoritmu \ref{alg:largest_submatrix}: 
V prvom prechode maticou si u každého vrcholu zapamätáme počet jedničiek naľavo od neho, vratane daneho vrcholu. 
Tento prechod trvá lineárny čas.

V druhom prechode treba prejsť zaradom všetky stĺpce zľava doprava



\begin{algorithm}
\caption{Nájdenie najväčšej jednotkovej podmatice v matici  $m$x$n$}
\label{alg:largest_submatrix}
\begin{algorithmic}[1] % number one = line numbering is on
\REQUIRE matica $M$ rozmerov $m x n$ nad telesom $\Z_2$
\ENSURE pravý dolný roh podmatice aj s jej rozmermi


\FORALL {prvok $p$, $p \in M$}
	\IF {$p = 0$}
		\STATE nalavoOdPrvku(p) $\leftarrow$ 0
	\ELSE
		\STATE nalavoOdPrvku(p) $\leftarrow$ najdlhšia súvislá postupnosť jednotiek končiaca prvkom $p$
	\ENDIF	
\ENDFOR

\FOR {stĺpec $s$, $s \in M$ }
	\STATE Vytvor nový zásobník dvojíc (riadok, nalavoOdPrvku)
	\FOR {riadok $r$, $r \in M$ }
		\STATE $p \leftarrow (s, r)$
		
		\WHILE {Zásobník je neprázdny}
			\STATE vyberiem prvok $top$ z vrcholu zásobníka
			\IF {$top$ . nalavoOdPrvku $>$ nalavoOdPrvku(p)}
						
				\STATE prvokZoZasobnika $\leftarrow$ ($top$.r, s)
				\STATE DlzkaSekvencie(prvokZoZasobnika) = r - prvokZoZasobnika.r
			
			\ELSE
				\STATE Vložím prvok $prvokZoZasobnika$ do zásobníka
				\STATE break
			\ENDIF
		\ENDWHILE
		
		\STATE Do zásobníka vložím dvojicu (r, nalavoOdPrvku(p))
	\ENDFOR
	
	\WHILE {Zásobník je neprázdny}
		\STATE vyberiem prvok $top$ z vrcholu zásobníka
		\STATE prvokZoZasobnika $\leftarrow$ ($top$.r, s)
		\STATE DlzkaSekvencie(prvokZoZasobnika) = m - prvokZoZasobnika.r
		
	\ENDWHILE
\ENDFOR




\end{algorithmic}
\end{algorithm}
