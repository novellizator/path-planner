\chapter{T-maps --- programátorská dokumentácia}
\label{programdoc}

\section{Architektúra aplikácie}
Architektúra programu \emph{T-maps} je inšpirovaná konceptom MVC \cite{krasner_mvc_1988}.
Model predstavujú tvoria dve triedy: Map a CachedBitmap.
View a Controller sú spojené triede Form1.


Jednou zo schopností programu je nekonečný zoom. Ten je riešený TODO?? 
\begin{figure}[h]
\begin{lstlisting}[language=C++]
class CachedBitmap
{
    Map map;
    public void setMap(Map m);
    public void DrawBitmapInto(Graphics g, Point TLPoint, Size ViewPortSize, int squareS, bool isBichrom, bool forcePrecomputing = false);
 
    private void PrecomputeBitmap(Point TLPoint, Size viewPortSize);
}
\end{lstlisting}
\caption{Kľúčové funkcie triedy CachedBitmap}
\label{fig:cachedbitmap_interface_plus}
\end{figure}


\begin{figure}[h]
\begin{lstlisting}[language=C++]
class Map
{
    char[][] data;
    char[][] map;
    public void Load(string filename);
    public void LoadData(string filename);
}
\end{lstlisting}
\caption{Kľúčové funkcie triedy Map}
\label{fig:map_interface}
\end{figure}