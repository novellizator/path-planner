\chapter{T-maps --- programátorská dokumentácia}
\label{programdoc}

\section{Architektúra aplikácie}
Architektúra programu \emph{T-maps} je inšpirovaná konceptom MVC \cite{krasner_mvc_1988}.
Model tvoria dve triedy: Map (Obr.~\ref{fig:map_interface}) a CachedBitmap (Obr.~\ref{fig:cachedbitmap_interface_plus}).
View a Controller sú spojené triede Form1.


Aplikácia bola navrhnutá s dôrazom na nízke systémové nároky aj pri operáciách, ako je nekonečný zoom. 
Ten je riešený tak, že v triede $CachedBitmap$  je uložená bitmapa odpovedajúca výseku znakovej matice, ktorá sa v prípade potreby prepočíta.
Rozmery tejto bitmapy odpovedajú dvojnásobku rozmerov zobrazovacieho okna aplikácie.

\section{Kľúčové funkcie aplikácie}



\begin{figure}[H]
\begin{lstlisting}[language=C++]
class Map
{
    char[][] data;
    char[][] map;
    public void Load(string filename);
    public void LoadData(string filename);
}
\end{lstlisting}
\caption{Kľúčové funkcie triedy Map. Obsahuje dvojrozmerné polia reprezentujúce znakovú maticu a taktiež dáta z dátového súboru}
\label{fig:map_interface}
\end{figure}



\begin{figure}[H]
\begin{lstlisting}[language=C++]
class CachedBitmap
{
    private Bitmap cachedBitmap;
    private bool isBichromatic;
	
	// dalsie premenne viazane na vlastnosti cachedBitmap
	// ...
	
    private Map map;
    public void setMap(Map m);
    public void DrawBitmapInto(Graphics g, Point TLPoint, Size ViewPortSize, int squareS, bool isBichrom, bool forcePrecomputing = false);
 
    private void PrecomputeBitmap(Point TLPoint, Size viewPortSize);
}
\end{lstlisting}
\caption{Kľúčové funkcie triedy CachedBitmap. 
Obsahuje v sebe referenciu na triedu Map a úsek veľkej mapy cachedBitmap. Jej hlavnou metódou je DrawBitmapInto,
ktorej parametry špecifikujú úsek, ktorý sa má vykresliť. Metóda ho vykreslí pomocou grafického objektu $g$ a v prípade nutnosti zavolá funkciu
PrecomputeBitmap na predpočítanie mapy.}
\label{fig:cachedbitmap_interface_plus}
\end{figure}
