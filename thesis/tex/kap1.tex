\chapter{Zadanie problému a cieľové požiadavky}

\section{Úvodné definície a značenia}
Na začiatok si zaveďme niektoré dôležité pojmy z teórie grafov.
Úlohu so všetkými jej špecifikami si ozrejmíme v nasledujúcich podkapitolách.
\begin{define}
{\sl Graf G} je usporiadaná dvojica (V, E), kde V označuje množinu vrcholov(vertices) a $E \subseteq V \times V $ označuje množinu hrán (edges). Značíme G = (V, E).
\end{define}


\begin{note}
Hrana je jednoznačne určená dvojicou vrcholov.
\end{note}


\begin{define}
{\sl Ohodnotený graf (G, w)} je graf G~spolu s~reálnou funkciou (tzv. ohodnotením)
$w: E(G) \to \R$, kde $w$ je funkcia, ktorá každej hrane priradí
reálne číslo reprezentujúce \emph{dĺžku}, alebo \emph{hodnotu} hrany.
\end{define}

Ukážka obecného ohodnoteného grafu je na obrázku \ref{fig:ohodnoteny_graf}.

\begin{figure}[h]
\centering
\includegraphics[height=5.5cm]{./img/graf.eps}
\caption{Ohodnotený graf}
\label{fig:ohodnoteny_graf}
\end{figure}

Pri hľadaní najkratšej cesty v grafe pracujeme s pojmamy, ako 
sú \emph{cesta} a \emph{najkratšia cesta}.

\begin{define}
{\sl Cesta P z vrcholu $v_0$ do vrcholu $v_n$ v grafe G } je postupnosť $P = (v_{0},e_{1},v_{1},\dots, e_{n}, v_{n})$,
pre ktorú platí $e_{i} = \{v_{i-i},v_{i}\}$ a taktiež
$v_{i} \ne v_{j}$ pre každé $i \ne j$.
\end{define}

Všimnime si, že na ceste nenavštívime žiaden vrchol dvakrát a teda cesta neobsahuje kružnice.


\begin{define}
{\sl Dĺžka cesty P z vrcholu $v_0$ do vrcholu $v_n$ v ohodnotenom grafe (G, w) } je súčet dĺžok hrán, ktoré sa na~ceste nachádzajú.
\end{define}

Samozrejme, medzi dvoma vrcholmi môže existovať viacero ciest.

\begin{define}
{\sl Najkratšia cesta P z vrcholu $v_0$ do vrcholu $v_n$
v~ohodnotenom grafe (G, w)} 
je cesta z~vrcholu $v_0$ do vrcholu $v_n$ s najmenšou dĺžkou. 
\end{define}


\section{Mriežkový graf}

Keď sme si už zaviedli kľúčové pojmy, prejdime k samotnému
zadaniu úlohy.
Ako sme už spomínali, problém budeme riešit na tzv. mriežkových grafoch. Čo je mriežkový graf a v čom sa od obecného grafu odlišuje?

V hernej praxi predstavuje mriežkový graf mapu hracej plochy, s ktorou sa stretávame v najrôznejších hrách, ako je Warcraft, Startcraft, Dragon Age \cite{sturtevant2012benchmarks}
a~podobne.

Ide o špeciálny a dosť obmedzený typ grafu. Vizuálne si ho môžme predstaviť ako konečný graf v ktorom sú vrcholy rozostúpené v tvare mriežky a hrana
je stále medzi dvojicami susedných vrcholov vo všetkých ôsmych smeroch. Dĺžka vodorovnej alebo zvislej hrany je $1$ a dĺžka šikmej hrany je $\sqrt{2}$.

\begin{note}
Mriežkový graf patrí medzi {\sl riedke} grafy,
pretože má veľmi malý počet hrán (lineárny od počtu vrcholov).
\end{note}


\begin{figure}[h]
\centering
\includegraphics[height=3.5cm]{./img/mriezkovy_graf2x2.eps}
\caption{Mriežkový graf 2x2}
\label{fig:mriezkovy_graf2x2}
\end{figure}


\begin{figure}[h]
\centering
\includegraphics[height=5.5cm]{./img/mriezkovy_graf.eps}
\caption{Mriežkový graf bez označenia vrcholov a dĺžok hrán}
\label{fig:mriezkovy_graf}
\end{figure}


Zadefinujme si teraz mriežkový graf formálne.

\begin{define}
{\sl Mriežkový graf rozmerov $m \times n$} je ohodnotený graf s~ohodnotením $w$ s $m \cdot n$ vrcholmi očíslovanými od $v_{1,1}$ až po $v_{m,n}$ 
s~priamymi hranami $j$ v~tvare $\{v_{a,b}, v_{a,b+1}\}, \{v_{a,b}, v_{a+1,b}\}$, kde $w(j) = 1$ 
a šikmými hranami $ s $ v~tvare 
$\{v_{a,b}, v_{a+1,b+1}\}$, $\{v_{a,b}, v_{a-1,b+1}\}$, kde $ w(s) = \sqrt{2}$.
\end{define}

\begin{note}
	Mriežkový graf sa dá reprezentovať ako matica $m \times n$ nad telesom $\Z_2$, kde jednotky predstavujú miesta, kde sú vrcholy. 
\end{note}

\begin{note}
\label{note:popis_vrcholu}
	Konkrétny vrchol $v_{x,y}$ sa dá taktiež popísať jedným čislom $c:= x\cdot n + y$, kde $n$ je počet stĺpcov matice.
\end{note}


\begin{example}
Mriežkový graf rozmerov $2 \times 2$ s vyznačenými dĺžkami
hrán vidíme na obrázku~\ref{fig:mriezkovy_graf2x2}.
Príklad mriežkového grafu $4 \times 7$ je na obrázku~\ref{fig:mriezkovy_graf}.
Príklad jeho maticovej reprezentácie je na obrázku~\ref{fig:maticova_reprezentacia}.

\end{example}


\begin{figure}[h]


\[
G =
  \begin{bmatrix}
    1 & 0 & 1 & 1 & 1 & 1 & 0\\
	1 & 1 & 1 & 0 & 1 & 0 & 1\\
	1 & 1 & 1 & 1 & 0 & 0 & 1\\
	1 & 1 & 0 & 1 & 1 & 1 & 1\\
  \end{bmatrix}
\]

\caption{Maticová reprezentácia mriežkového grafu}
\label{fig:maticova_reprezentacia}
\end{figure}




\section{GPPC: Grid-Based Path Planning Competition}
Algoritmus navrhnutý a naprogramovaný v tejto práci bol zaradený do súťaže \textbf{GPPC}, ktorá sa koná približne raz ročne.

\subsection{Špecifiká súťaže, limity}

Mriežkové grafy budú mať rozmery maximálne $2048 \times 2048$.
Súťaž bude rozdelená do dvoch fáz --- fázy predspracovania grafu (pre-processing)
a fázy testovania. Na predspracovanie grafu bude vyhradený čas
maximálne 30 minút a program si svoje dáta uloží na disk do súboru o veľkosti maximálne 50MB.
Potom vo fáze testovania budé dostávať požiadavky na nájdenie najkratšej cesty. Úlohou je naimplementovať interface zobrazený na obrázku \ref{fig:interface}. 

\begin{figure}
\begin{lstlisting}[language=C++]
struct xyLoc 
{
    int x;
    int y;
};

void PreprocessMap(std::vector<bool> &bits, int width, int height, const char *filename);
                   
void *PrepareForSearch(std::vector<bool> &bits, int width, int height,const char *filename);

bool GetPath(void *data, xyLoc s, xyLoc g, std::vector &path);
\end{lstlisting}
\caption{Interface, ktorý treba naimplementovať}
\label{fig:interface}
\end{figure}
Funkcia \emph{PreprocessMap} má za úlohu predspracovať
mriežkový graf a vytvoriť si pomocné dátové štruktúry podľa predpísaných limitov. Jej prvým argumentom 
je samotný mriežkový graf predaný ako zlinearizovaná matica, ktorú bolo vidieť na obrázku~\ref{fig:maticova_reprezentacia}. Ďalšími argumentami sú rozmery grafu,
keďže z jednorozmernej reprezentácie nie sú odvoditeľné. Posledným argumentom je názov súboru, do ktorého sa pomocné dátové štruktúry uložia.

Funkcia \emph{PrepareForSearch} slúži na načítanie dát z vytvoreného súboru. Argumenty má rovnaké, ako predošlá funkcia. Štvrtý argument
dodáva názov súboru, z ktorého sa pomocné dáta načítajú. Funkcia po načítaní dát skonštruuje dátové štruktúry a vráti na nich smerník.

Tento smerník prevezme v prvom argumente funkcia \emph{GetPath}, ktorá slúži na nájdenie najkratšej cesty. Jej ďalšími argumentami
sú súradnice počiatočného a koncového bodu, pre ktoré sa bude hľadať cesta a posleným argumentom je referencia na vektor, do ktorého sa samotná cesta uloží. Funkcia vracia hodnotu typu boolean na základe toho, či už vypočítala istý úsek a chce vrátiť cestu. Zíde sa napríklad, keď treba vrátiť čo najrýchlejšie aspoň prvých $k$ krokov cesty (viď.~Kritériá, sekcia~\ref{kriteria}).



\subsection{Kritériá súťaže, hodnotenie programov}
Na algoritmus môžeme klásť rôzne požiadavky, ktoré si častokrát navzájom protirečia, preto programy posudzujeme poďla viacerých kritérií.
\label{kriteria}
\begin{itemize}
\item Celkový čas na~nájdenie cesty.
\item Čas na nájdenie prvých 20 tich krokov.
\item Dĺžka cesty (zohľadnená suboptimalita).
\item Maximálny čas vrátenia hociktorej časti cesty.
\end{itemize}

Testovací počítač má 12 GB RAM pamäti a dva 2.4 GHz Intel Xeon E5620
procesory.