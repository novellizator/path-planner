\chapter*{Úvod}
\addcontentsline{toc}{chapter}{Úvod}

Slávna Eulerova úloha siedmych mostov v Kaliningrade \cite{euler41} sa považuje za prvú prácu, 
ktorá zaviedla teóriu grafov.
Úlohou je prejsť po týchto siedmych mostoch tak, aby sme po každom prešli práve raz.


\begin{figure}[h]
\centering
\includegraphics[height=7.5cm]{./img/Konigsberg_bridges.png}
\caption{Sedem mostov v Kaliningrade, \url{http://en.wikipedia.org/wiki/Seven_Bridges_of_K\%C3\%B6nigsberg}}
\label{fig:konigsberg_bridges}
\end{figure}

 
Od tej doby sa využitie teórie grafov značne
rozšírilo a v dnešnej dobe patrí medzi významné
a rozpracované teórie. V modernej dobe je jedným z jej 
naj\-dô\-le\-ži\-tej\-ších 
problémov hľadanie najkratšej cesty. Najčastejšie sa s~nimi stretávame pri plánovaní trasy v~GPS navigácii.
Medzi najvýznamnešie práce považujeme práce od Dijkstru \cite{dijkstra59} a Floyd-Warshalla \cite{floyd62}.

So~začiatkom fenoménu počítačových hier 
a umelej inteligencie sa do povedomia dostal špeciálny typ grafu --
mriežkový graf, využívaný ako herná mapa.
V~hrách často trebalo nájsť cestu pre počítačom
ovládanú postavičku z miesta A do miesta B.
Nakoľko je väčšina hier komerčná, algoritmy
využívané v~hrách boli a~sú taktiež komerčné.
Dôsledkom toho nie sú verejne publikované a~porovnané rôzne prístupy a~algoritmy
na~vyhľadávanie najkratších ciest v~mriežkových mapách. A~keď už aj sú, tak práce používajú rôzne mapy
na~bechmarking a~teda neexistuje žiadna globálna porovnávacia štúdia týchto prístupov.

Súťaž {\sl Grid-Based Path Planning Competition}
 \cite{sturtevantgppc} sa snaží tento problém vyriešiť tým, že porovnáva rôzne algoritmy na~veľkej množine máp
použitých v~známych počítačových hrách a~vyhodnocuje ich úspešnosť v~rámci viacerých kategórií.

Cieľom tejto práce je spraviť prehľad doterajších prístupov k~tomuto problému a~prispieť vlastným algoritmom do sútaže a~niekoľkými vylepšeniami k~doterajším prístupom hľadania najkratšej cesty na~mriežkových grafoch.

V prvej kapitole si zavedieme kľúčové termíny a~popíšeme problém formálne. Na~konci kapitoly spomenieme súťaž, ktorej sa daný algoritmus zúčastnil.
Druhá kapitola je zameraná na~vytvorenie prehľadu kľúčových algoritmov použivaných na~riešenie problému.
V~ďalšej kapitole navrhneme vlastné riešenie založené na~poznatkoch popísaných v~druhej kapitole s~pridaním vlastných vylepšení.
A v poslednej kapitole toto riešenie porovnáme s dosavadnými.

Práca obsahuje dve prílohy: stručnú užívateľskú a programátorskú do\-ku\-men\-tá\-ciu programu \emph{T-maps}, ktorý slúži na grafické zobrazenie mriežkového grafu.

