\chapter*{Úvod}
\addcontentsline{toc}{chapter}{Úvod}
\cite{sturtevant2012benchmarks}
Slávna Eulerova úloha siedmych mostov v Kaliningrade \cite{euler41} sa považuje za prvú prácu, 
ktorá zaviedla teóriu grafov. Úlohou je prejsť po týchto siedmych mostoch tak, aby sme po každom šlo práve raz.
Od tej doby sa využitie teórie grafov značne
rozšírilo a v dnešnej dobe patrí medzi významné
a rozpracované teórie. V modernej dobe je jedným z jej 
najdôležitejších problémov hľadanie najkratšej cesty. Najčastejšie sa s nimi stretávame pri 
GPS navigacii.
Medzi najvýznamnešie práce považujeme práce od Dijkstru \cite{dijkstra59} a Floyd-Warshalla.

S narastajúcim fenoménom počítačových hier 
a umelej inteligencie sa do povedomia dostal špeciálny typ grafu --
mriežkový graf, využívaný ako herná mapa.
V hrách trebalo často nájsť cestu pre počítačom
ovládanú postavičku z miesta A do miesta B.
Nakoľko je ale väčšina hier komerčná, algoritmy
využívané v hrách boli a sú taktiež komerčne.
Dôsledkom toho nie sú publikované a porovnané rôzne prístupy a algoritmy
na vyhľadávanie v mriežkových mapách. A keď už aj sú, tak práce používajú rôzne mapy
na bechmarking a teda neexistuje žiadna globálna porovnávacia štúdia týchto prístupov.
Súťaž {\sl Grid-Based Path Planning Competition}
 \cite{sturtevantgppc} sa snaží tento problém vyriešiť tým, že ktorá porovnáva rôzne algoritmy na veľkej množine máp
použitých v známych počítačových hrách.

Cieľom tejto práce je spraviť prehľad doterajších prístupov k tomuto problému a prispieť vlastným algoritmom
do súťaže.




V práci sme naimplementovali vlastný algoritmus a porovnali ho s doterajšímy známymi.
TODO?? Počas práce sme prišli na zaujímavé zefektívnenie algoritmov [snad na nieco prijdem :))]a dúfame v jeho rozšírenie do hernej sféry.

ASK?? ake su vlastne ciele? mam vymysliet vzbrusu novy algoritmus?

V prvej kapitole si zadefinujeme kľúčové termíny a popíšeme problem. Na konci kapitoly spomenieme súťaž, ktorej sa daný algoritmus zúčastnil 
a popíšeme jej podmienky.
Druhá kapitola sa pokúsime rozobrať doterajšie zistenia a algoritmy používané na riešenie obdobných problémov.
V tretej kapitole popíšeme náš algoritmus a vo štvrtej kapitole ho porovnáme s ostatnými algoritmami a uvedieme výsledky.
