\chapter{Prehľad algoritmov}
Na hľadanie najkratších ciest v grafe poznáme mnoho algoritmov, ktoré vieme rozdeliť do troch skupín.


\begin{itemize}
\item Point To Point Shortest Path - hľadajú najkratšiu cestu medzi dvoma zadanými bodmi
\item Single Source Shortest Path - pre daný vrchol {\sl v} hľadajú najkratšiu cestu do všetkých vrcholov grafu.
\item All Pairs Shortest Path - skúmajú najkratšiu cestu medzi všetkými dvojicami vrcholov.
\end{itemize}

Napriek tomu, že sú tieto problémy na obecných grafoch NP-ťažké, na mriežkových grafoch, kde majú všetky vrcholy kladnú cenu, vieme nájsť riešenie v polynomiálnom čase.
V práci sa budeme ďalej zaoberať riešením prvého problému (Point to Point Shortest Path).

V tejto kapitole si popíšeme algoritmy, ktoré sú použiteľné na všetkých grafoch 
s nezápornými dĺžkami hrán.

\section{Dijkstrov algoritmus}
Medzi základné algoritmy patrí Dijkstrov algoritmus, ktorý je asymptoticky optimálny (TODO?? for sure?).

Pri hľadaní cesty z vrcholu $s$ do vrcholu $t$ prechádzame postupne vrcholy zo stúpajúcou vzdialenosťou od $s$, až dokým sa nedostaneme k cieľovému vrcholu $t$.
Vizuálne si beh algoritmu môžme predstaviť ako kruh so stredom v bode $s$ so zväčšujúcim sa polomerom. Algoritmus napísaný v pseudokóde je nasledovný:


\begin{algorithm}
\caption{Dijkstra: Nájdi najkratšiu cestu medzi dvoma bodmi {\sl s} a {\sl t}}
\label{alg:dijkstra}
\begin{algorithmic}[1] % number one = line numbering is on
\REQUIRE $s=(x_s,y_s), t=(x_t,y_t)$
\ENSURE $path$


\STATE path.append($(x_s, y_s)$)
\COMMENT {pridám počiatok}

\WHILE {$x_s \neq x_t \vee y_s \neq y_t $}
	\IF {$x_s \textless x_t$}
		\STATE $x_s \leftarrow x_s + 1$
	\ELSIF {$x_s \textgreater x_t$}
		\STATE $x_s \leftarrow x_s - 1$
	\ENDIF

	\IF {$y_s \textless y_t$}
		\STATE $y_s \leftarrow y_s + 1$
	\ELSIF {$y_s \textgreater y_t$}
		\STATE $y_s \leftarrow y_s - 1$
	\ENDIF
	\STATE path.append($(x_s, y_s)$)
\ENDWHILE

\end{algorithmic}
\end{algorithm}



\section{Lineárny Dijkstrov algoritmus}
Môž


\section{A*}
dalsi algoritmus do zbierky je a* \cite{astar72}.
