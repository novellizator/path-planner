\chapter{Testovanie a výsledky}

\section{Kritériá a popis testovania}
\subsection{Vstupné dáta}
Častým problémom pri vzájomnom porovnávaní algoritmov je
nájsť testovaciu vzorku, ktorá by otestovala beh algoritmu na širokej škále grafov.
Ako bolo v úvode spomenuté, súťaž {\sl Grid-Based Path Planning Competition} poskytuje množstvo máp rôznych typov a rozmerov,
na ktorých sa algoritmy dajú testovať. Formát máp popisuje článok \cite{sturtevant2012benchmarks}.

Programy boli testované na troch mapách. Ich grafická reprezentácia bola 
vytvorená pomocou programu T-map (popis programu a významu farieb sa nachádza v prílohe \ref{userdoc}) 
a~môžme ju vidieť na obrázkoch Obr.~\ref{fig:aftershock_map},~Obr.~\ref{fig:brushfire_map} a Obr.~\ref{fig:biggamehunters_map}.
Pripomeňme si, že žltou farbou je označená priechodná oblasť; ostatné oblasti sú nepriechodné.

Každá mapa bola testovaná oproti dvom množinám vstupov.
Každá z týchto množín obsahuje približne 200 dopytov
 na~najkratšiu vzdialenosť medzi dvojicou zadaných bodov.
Jedna z tých dvoch množín obsahuje dvojice bodov, ktoré ležia relatívne \uv{blízko} seba, druhá množina obsahuje dvojice bodov, medzi ktorými je vzdialenosť väčšia.


Konkrétne máme tieto 4 vstupné súbory:
\begin{itemize}
\item AS --- vstupný súbor k mape Aftershock obsahujúci \uv{blízke} body.
\item AL --- vstupný súbor k mape Aftershock obsahujúci vzdialenejšie body.
\item BS --- vstupný súbor k mape Brushfire obsahujúci \uv{blízke} body.
\item BL --- vstupný súbor k mape Brushfire obsahujúci vzdialenejšie body.
\item GS --- vstupný súbor k mape BigGameHunters obsahujúci \uv{blízke} body.
\item GL --- vstupný súbor k mape BigGameHunters obsahujúci vzdialenejšie body.
\end{itemize}


\begin{figure}[H]
\centering
\includegraphics[width=10cm]{./img/Aftershock.png}
\caption{Graf \uv{Aftershock} má rozmery $512 \times 512$}
\label{fig:aftershock_map}
\end{figure}

\begin{figure}[H]
\centering
\includegraphics[width=10cm]{./img/Brushfire.png}
\caption{Graf \uv{Brushfire} má rozmery $512 \times 512$}
\label{fig:brushfire_map}
\end{figure}

\begin{figure}[h]
\centering
\includegraphics[width=10cm]{./img/BigGameHunters.png}
\caption{Graf \uv{BigGameHunters} má rozmery $512 \times 512$}
\label{fig:biggamehunters_map}
\end{figure}

\subsection{Testovacie kritériá}
Algoritmy testujeme podľa nasledovných kritérií:

\begin{itemize}
\item Rýchlosť nájdenia cesty.
\item Dĺžka trasy.
\item Počet prehľadaných vrcholov.
\end{itemize}


\subsection{Testované algoritmy}
Testovať budeme nasledujúce algoritmy:
\begin{itemize}
\item Dijkstrov algoritmus nad binárnou haldou (označíme DH).
\item Dijkstrov algoritmus nad priehradkovou haldou (označime DBQ).
\item A* používajúci 6 landmarkov (označíme A*).
\item A* používajúci 6 landmarkov a taktiež obdĺžnikovú dekompozíciu, popísanú v kapitole 3 (označíme A*+).
\item Anderson -- s prehľadom najrýchlejší algoritmus, ktorý vyhral súťaž (oz\-na\-čí\-me And).
\end{itemize}


\subsection{Kompilácia}
Kód bude kompilovaný kompilátorom g++ s direktívou \emph{-O3}, čo zaručuje maximálnu rýchlosť a efektivitu behu programu.


\section{Výsledky}
Tabuľka \ref{fig:totaltime_result} popisuje čas v sekundách, ktorý strávil daný algoritmus hľadaním cesty medzi všetkými dvojicami daného vstupu. Z tabuľky môžme vypozorovať  zaujímavé správanie algoritmov. Najrýchlejší je 
algoritmus Anderson, ktorý je zatiaľ najrýchlejším algoritmom v tejto súťaži.
Ďalej vidíme, že dijkstrov algoritmus je na niektorých mapách a vstupoch niekoľkonásobne rýchlejší ako A* a~na~niektorých zasa pomalší.
Dekompozícia mapy na~obdĺžniky zrýchlila algoritmus A* len keď bola vzdialenosť medzi počiatočným a koncovým bodom dostatočne malá.



\begin{table}[H]
	\centering
	\begin{tabular}{|l | r|r|r|r|r|r|}
	\hline
	  & AS & AL & BS & BL & GS & GL \\
	\hline
	DH & 0.372839 & 9.978722 & 0.356326 & 5.425334 & 0.485236 & 14.354057 \\
	DBQ & 0.506893 & 11.505914 &  0.329444 & 4.315670 & 0.467251 & 11.990992 \\
	A* & 0.263951 & 7.481771 & 0.283890 & 15.599357 & 0.393476 & 9.667814 \\
	A*+  & 0.182731 & 7.502437 &  0.197504 & 15.670339 & 0.257229 & 9.694499\\
	And  &  0.000543 & 0.002659 & 0.000359 & 0.001747 & 0.000673& 0.002197\\
	\hline
	\end{tabular}
	\caption{Časy behu algoritmov nad zadanými množinami vstupov. Uvedené v sekundách}
	\label{fig:totaltime_result}
\end{table}

Tabuľka~\ref{fig:totalpath_result} uvádza súčet dĺžok najkratších ciest pre danú testovanú množinu. Vyplýva z~nej, že rýchly algoritmus Anderson zďaleka nie je optimálny a táto neoptimalita sa prejavuje najmä keď je vzdialenosť medzi počiatkom a koncom malá. Ostatné algoritmy vykazujú podobne dobré výsledky.
\begin{table}[H]
	\centering
	\begin{tabular}{|l | r|r|r|r|r|r|}
	\hline
	 & AS & AL & BS & BL & GS & GL \\
	\hline
	DH & 8710.17 & 144046 & 8731.65& 152185 & 8744.62 & 128398 \\
	DBQ & 8710.17 & 144046 & 8731.65 & 152185 & 8744.62 & 128398 \\
	A* & 8743.7 & 144527& 8762.74 & 153288 & 8774.98 & 128885 \\
	A*+ & 8738.43& 144527 & 8758.98 & 153288 & 8769.46 & 128885 \\
	And & 22057.2 & 144854 & 22010 & 156927 & 27717.8 & 135411 \\
	\hline
	\end{tabular}
	\caption{Súčet dĺžok ciest pre dané dvojice počiatkov a koncov}
	\label{fig:totalpath_result}
\end{table}

Posledná tabuľka~\ref{fig:verticesscanned_result} popisuje
počet vrcholov grafu, ktoré algoritmus prehľadal, ked hľadal najkratšiu cestu. Táto veličina bola zisťovaná len z algoritmov autora. Z tabuľky vyplýva, že algoritmus A* prehľadáva menej vrcholov, ako dijkstrov algoritmus. Taktiež z nej vyplýva, že dekompozícia grafu na obdĺžniky je pri vzdialených vrcholoch zbytočná.
\begin{table}[H]
	\centering
	
	\begin{tabular}{|l | r|r|r|r|r|r|}
		\hline
		  & AS & AL & BS & BL & GS & GL \\
		\hline
		DH & 1067195 & 35126970 & 926032 & 19103546 & 986303 & 35325535 \\
		DBQ & 1347329 & 39028967 & 1176089 & 20528587 & 1255147 & 40707226 \\
		A* & 350839 & 13854096& 624561 & 41883515 & 637189 & 17936238 \\
		A*+ & 250022& 13854096 & 465311 & 41883515 & 407731 & 17936238 \\

		\hline
	\end{tabular}
		
	\caption{Výsledky obsahujúce počet navštívených vrcholov}
	\label{fig:verticesscanned_result}
\end{table}


