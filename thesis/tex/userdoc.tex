\chapter{T-maps --- užívateľská dokumentácia}
\label{userdoc}
\section{Popis programu T-maps}
Program T-maps slúži na grafické znázornenie mriežkového grafu, najkratších ciest a prehľadávaných vrcholov.
Mriežkový graf je interpretovaný ako farebná bitmapa, kde rôzne farby označujú rôzne typy vrcholov.

Typy vrcholov a ich farba: TODO? vrcholov? su len vrcholy a nevrcholy...
\begin{itemize}
\item Strom - zelená.
\item Voda - modrá.
\item Bažina a defaultná priechodná oblast - papájovožltá.
\item Defaultná nepriechodná oblasť - sivá.

\end{itemize}

Pokiaľ bol vrchol prehľadávaný počas hľadania najkratšej cesty, jeho farba bude oranžová. Ak je dokonca súčasťou najkratšej cesty, 
tak bude červený.

\section{Použitie programu}

TODO??-- obrazky = program s mapou, bichromaticka mapa teda graf, priblizenie detailu(dodaj infinite zoom), nacitana najkratsia cesta s prehladanymi vrcholmi
