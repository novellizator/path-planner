\chapter{T-maps --- užívateľská dokumentácia}
\label{userdoc}
\section{Popis programu T-maps}
Program T-maps načíta mriežkový graf vo forme znakovej matice a vykreslí ho. K danému grafu môže voliteľne načítať dátový súbor, ktorý obsahuje
informácie o~navštívených vrcholoch a~o~najkratšej ceste.
Túto maticu aj s~dátami vykreslí na~obrazovku. Výstup pripomína mapu používanú v~počítačových hrách.
Zvláda niekoľko ďalších operácií, ako je priblíženie a~oddialenie mapy, posúvanie sa po~mape a~uloženie aktuálne prehliadaného úseku mapy.


\section{Formát vstupu a výstupu}
Presný formát súboru s~mriežkovým grafom je popísaný v~\cite{sturtevant2012benchmarks} a~je zhodný s~formátom, 
ktorý používa algoritmus.

Dátový súbor obsahuje dva riadky. Prvý riadok obsahuje medzerou oddelené čísla vrcholov, ktoré ležia na najkratšej ceste.
Druhý riadok obsahuje čísla vrcholov, ktoré boli navštívené počas behu algoritmu. ASK?? navstivene je ok??
Pokiaľ je vrchol v $r$-tom riadku matice a $s$-tom stĺpci. Pričom širka matice je označená ako $sirka$. TODO?? mozno supnut do prvej kapitoly...
 Potom číslo vrcholu sa vypočita podľa vzorca $cisloVrcholu := r*sirka + s$. 

Výstup môže byť exportovaný do viacerých grafických formátov, medzi ktorými sú GIF, PNG, BMP a JPEG.



Typy vrcholov a ich farba: TODO? vrcholov? su len vrcholy a nevrcholy...
\begin{itemize}
\item Strom - zelená.
\item Voda - modrá.
\item Bažina a defaultná priechodná oblast - papájovožltá.
\item Defaultná nepriechodná oblasť - sivá.

\end{itemize}

Pokiaľ bol vrchol prehľadávaný počas hľadania najkratšej cesty, jeho farba bude oranžová. Ak je dokonca súčasťou najkratšej cesty, 
tak bude červený.

\section{Použitie programu}

TODO??-- obrazky = program s mapou, bichromaticka mapa teda graf, priblizenie detailu(dodaj infinite zoom), nacitana najkratsia cesta s prehladanymi vrcholmi
